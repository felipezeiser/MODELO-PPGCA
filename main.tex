\documentclass[oneside,brazilian,english]{conf/UNISINOSmonografia}
\usepackage[utf8]{inputenc} 
\usepackage[T1]{fontenc} 
\usepackage{graphicx}

\usepackage{tcolorbox}
\usepackage{amsfonts}

\usepackage{xspace}
\usepackage{multirow}

\usepackage{longtable}
\usepackage{pdflscape}
\usepackage[titles]{tocloft}

\usepackage{arydshln}


\tolerance=1
\emergencystretch=\maxdimen
\hyphenpenalty=10000
\hbadness=10000


\usepackage{hyperref}
\hypersetup{
    colorlinks,
    citecolor=black,
    filecolor=black,
    linkcolor=black,
    urlcolor=black
}



\usepackage[alf, 	
			 		abnt-emphasize=bf,
				    abnt-url-package=none,
				    abnt-repeated-title-omit=yes,
				    abnt-full-initials=yes, %yes nome por extenso, no apenas iniciais
				    bibjustif,
					abnt-etal-list=3												%abreviar com mais de 3 autores
]{abntex2/abntex2cite}

\usepackage{pdfpages}

\titulo{Modelo de Dissertação e Tese - Programa de Pós-graduação em Computação Aplicada Unisinos}
\autor{Sobrenome}{Nome}

\orientador[Prof.]{Sobrenome, ~PhD}{Nome Orientador}
\coorientador[Prof.]{Sobrenome, ~PhD}{Nome do Coorientandor}

\local{São Leopoldo}
\ano{2020}

\unidade{Unidade Acadêmica de Pesquisa e Pós-Graduação}
\curso{Programa de Pós-Graduação em Computação Aplicada}
\nivel{Nível Mestrado}
\natureza{
Master Thesis presented as a partial requirement to obtain the Master’s degree from the Applied Computing Graduate Program of the Universidade do Vale do Rio dos Sinos — UNISINOS}

\cip{Dissertação (mestrado)}{007.987}
\bibliotecario{Bibliotecária responsável: Fulana da Silva}{12/3456}

\palavrachave{english}{Keyword 1}
\palavrachave{english}{Keyword 2}
\palavrachave{english}{Keyword 3}

\palavrachave{brazilian}{Palavra chave 1}
\palavrachave{brazilian}{Palavra chave 2}
\palavrachave{brazilian}{Palavra chave 3}


\begin{document}

\capa
\folhaderosto
\folhadeaprovacao 


\begin{dedicatoria}
To my family.\\[4ex]
\end{dedicatoria}
\begin{dedicatoria}

\begin{itshape} 
"I think, therefore I am."\\
\end{itshape}
--- \textsc{Rene Descartes} % \textsc é o "small caps"
\end{dedicatoria}

%=======================================================================
% Agradecimentos (opcional).
%=======================================================================
\begin{agradecimentos}

AGRADECIMENTOS

\end{agradecimentos}

%=======================================================================
% Epígrafe (opcional).
%
% ``[...] o autor apresenta uma citação, seguida de indicação de autoria,
% relacionada com a matéria tratada no corpo do trabalho. Podem, também,
% constar epígrafes nas folhas de aberturas das seções primárias.''
%=======================================================================
%\begin{epigrafe}
%``\textit{Ninguém abre um livro sem que aprenda alguma coisa}''.\\
%(Anônimo)
%\end{epigrafe}

%=======================================================================
% Resumo.
%
% A recomendação é para 150 a 500 palavras.
%=======================================================================
\begin{abstract}

    \noindent Este documento apresenta orientações para uso da classe \LaTeX\ de formatação de dissertações e teses para o PPGCA UNISINOS\@.  Ao mesmo tempo, ele serve como exemplo de uso da classe, demonstrando os principais comandos a serem utilizados, e outras orientações mais gerais de uso do \LaTeX.  Adicionalmente, procuramos incluir no documento algumas orientações sobre a escrita da monografia em si, reunindo dicas e recomendações que contribuem para aumentar a qualidade técnica dos trabalhos acadêmicos.  O Resumo deve conter de 150 a 500~palavras e nele não deve haver citações. Sugere-se a utilização de parágrafo único.
    \end{abstract}

\begin{otherlanguage}{brazilian}
  \begin{abstract}
    \noindent Este documento apresenta orientações para uso da classe \LaTeX\ de formatação de dissertações e teses para o PPGCA UNISINOS\@.  Ao mesmo tempo, ele serve como exemplo de uso da classe, demonstrando os principais comandos a serem utilizados, e outras orientações mais gerais de uso do \LaTeX.  Adicionalmente, procuramos incluir no documento algumas orientações sobre a escrita da monografia em si, reunindo dicas e recomendações que contribuem para aumentar a qualidade técnica dos trabalhos acadêmicos.  O Resumo deve conter de 150 a 500~palavras e nele não deve haver citações. Sugere-se a utilização de parágrafo único.
  \end{abstract}
\end{otherlanguage}



%=======================================================================
% Lista de Figuras (opcional).
%=======================================================================
\setlength{\cftfignumwidth}{0.9in} % from p. 9
\renewcommand\cftfigindent{3pt}
\renewcommand\cftfigpresnum{\figurename{} }
\renewcommand\cftfigaftersnum{ -- }
\listoffigures


%=======================================================================
% Lista de Tabelas (opcional).
%=======================================================================
\setlength{\cfttabnumwidth}{0.9in} % from p. 9
\renewcommand\cfttabindent{3pt}
\renewcommand\cfttabpresnum{\tablename{} }
\renewcommand\cfttabaftersnum{ -- }
\listoftables

%=======================================================================
% Lista de Abreviaturas (opcional).
%
% Deve ser passada como parâmetro a maior das abreviaturas utilizadas.
%=======================================================================
%\begin{listadeabreviaturas}{seg., segs.}
%\item[ampl.] ampliado, -a
%\item[atual.] atualizado, -a
%\item[coord.] coordenador
%\item[N.~T.] Novo Testamento
%\item[seg., segs.] seguinte, -s
%\end{listadeabreviaturas}

%=======================================================================
% Lista de Siglas (opcional).
%
% Deve ser passada como parâmetro a maior das siglas utilizadas.
%=======================================================================
\begin{listadesiglas}{ABNT}
    \item[ABNT] Associação Brasileira de Normas Técnincas
\end{listadesiglas}

%=======================================================================
% Sumário
%=======================================================================
\renewcommand{\cftchapfont}{\bfseries\scshape}
\renewcommand{\cftsecfont}{\bfseries}
\tableofcontents

%=======================================================================
% Introdução
%=======================================================================
\chapter{Introduction}

% as epígrafes nos capítulos são opcionais
\epigrafecap{The reasonable man adapts himself to the world; the unreasonable one persists in trying to adapt the world to himself. Therefore all progress depends on the unreasonable man.}{George Bernard Shaw}

Conforme \citeonline{Hexsel11}, a introdução tem o objetivo de ``\emph{introduzir} o material que vai ser apresentado em mais detalhe nas seções subseqüentes''. Na introdução você deve contextualizar o problema e mostrar por que vale a pena resolvê-lo. Você deve apresentar a solução proposta e mostrar o seu diferencial em relação aos trabalhos relacionados. Observe, porém, que na introdução você deve apenas tratar do O QUÊ e PORQUÊ, sem tratar do como \cite{Hexsel11}, que deve ser explicado na seção que descreve o trabalho desenvolvido.

Geralmente, a introdução tem uma estrutura similar ao resumo e deve apresentar:
\begin{itemize}
	\item \textbf{Contexto e motivação:} Aqui você deve apresentar o contexto do trabalho (área de que ele se trata) e uma motivação para trabalhar nesse assunto.
	\item \textbf{Problema:} Aqui você vai apresentar um problema, uma lacuna, observada na área e que você pretende tratar. Você deve se perguntar aqui: ``Que respostas estou disposto a responder?''. O problema deve ser definido claramente e delimitado em termos de espaço de tempo. Veja que essa parte visa alertar o leitor de que o que você está propondo é uma solução para um problema observado na área. 
	\item \textbf{Objetivos:} Aqui você deve apresentar os objetivos do seu trabalho. Tome cuidado para não confundir objetivos com atividades.   Faça a si mesmo a pergunta: ``O que pretendo alcançar com a pesquisa?''. Você pode discernir entre objetivos gerais e objetivos específicos:
	\begin{itemize}
		\item Objetivo geral --- qual o propósito da pesquisa?
		\item Objetivos específicos --- abertura do objetivo geral em outros menores (possíveis capítulos).
	\end{itemize}
	Veja abaixo um exemplo de objetivo retirado da monografia de~\citeonline{Teixeira09}:

	Com a possibilidade de acesso a base de dados XML gerada a partir do Sistema de Currículos Lattes e a necessidade de melhor reutilizar as informações existentes neste sistema, o presente trabalho tem como objetivo geral permitir o acesso do pesquisador a seus dados através de uma interface mais amigável: o padrão LaTeX. Para isto destacam-se os seguintes objetivos específicos:
	\begin{alineas}
		\item identificar e analisar o formato de especificação de currículos da Plataforma Lattes;
		\item disponibilizar uma ferramenta para a geração de uma representação de dados intermediária a partir do formato especificado;
		\item implementar a tradução dos dados colhidos em código LaTeX através da utilização da ferramenta criada;
		\item analisar os resultados obtidos e as alternativas presentes no uso da ferramenta.
	\end{alineas}
\end{itemize}

\section{Comandos do \LaTeX}
Como regra geral, use os comandos tradicionais do \LaTeX\ para formatar seu texto.  Neste documento procuramos demonstrar os comandos mais comumente utilizados em monografias acadêmicas.

Neste capítulo apresentamos alguns exemplos de como colocar figuras e tabelas no seu texto.

\section{Ilustrações}

\subsection{Legendas}
As legendas das figuras devem se encontrar no topo da figura e não abaixo, como usualmente colocado. Abaixo da figura, é obrigatório colocar a fonte (mesmo que a figura tenha sido do próprio autor).

As legendas devem conter o tipo da ilustração (Figura, Tabela, etc), seguido de numeração simples (sem número do capítulo).

Toda figura deve ser citada no texto, como nos exemplos que seguem.

\subsection{Figuras}
A Figura~\ref{fig:escrita} ilustra as fases psicológicas da escrita da dissertação. Você vai se reconhecer no personagem. ;-)

\begin{figure}
	\caption{Fases psicológicas da escrita da dissertação}
	\label{fig:escrita}
	\centering%
	\begin{minipage}{.8\textwidth}
		\includegraphics[width=\textwidth]{figures/escrita.jpg}
		\fonte{\citeonline{Cham12}}
	\end{minipage}
\end{figure}

\subsection{Tabelas}
A Tabela~\ref{tab:estacoes} é um exemplo de tabela elaborada pelo(a) próprio(a) autor(a).

\begin{table}
	\caption{Período das estações do ano no Brasil}
	\label{tab:estacoes}
	\centering%
	\begin{minipage}{.6\textwidth}
		\begin{tabular*}{\textwidth}{ll}
			\hline
			\textbf{Meses} & \textbf{Estações do Ano}\\
			\hline
			21 de março a 21 de junho & Outono\\
			21 de junho a 23 de setembro & Inverno\\
			23 de setembro a 21 de dezembro & Primavera\\
			21 de dezembro a 21 de março & Verão\\
			\hline
		\end{tabular*}
		\fonte{Elaborado pelo autor.}
	\end{minipage}
\end{table}

\input{background.tex}
\input{relatedwork.tex}
\input{model.tex}
\input{methodology}
\input{results.tex}
\input{conclusion.tex}

\bibliography{bib/bibliography}

\end{document}
